\documentclass[letterpaper]{article}
\usepackage{graphicx}
\graphicspath{ {./images/} }
\usepackage{wrapfig}
\usepackage{xcolor}
\usepackage[top=1in, bottom=1.25in, left=1.00in, right=1.00in, landscape]{geometry}

\title{Payment Processing Methods}
\author{Hal DiMarchi}

\begin{document}
  \pagecolor{cyan!10}
  \begin{titlepage}
    \pagenumbering{gobble}
    \centering
    \vfill
    {\Large
        Sales Methods\\
        H. DiMarchi\\
    }
    \vfill
    \begin{figure}[h]
      \centering
      \includegraphics[width=12cm]{best-online-credit-card-processing}
    \end{figure}
  \end{titlepage}
  \tableofcontents
  \newpage
  \section{Introduction:}
    \paragraph{Webstore:}
    Our plan to release an online webstore for our customers put us into a new
    tier of business. Webstores offer convenience, and enhance
    our ability to engage high volumes of customers.
    \paragraph{Payment Methods:}
    A customer will want to be able to pay for any order made on our webstore. We will need to support
    credit card processing to allow this. Multiple payment methods are available. They must be evaluated
    in terms of ease of customer use, cost, security and, relatedly, PCI compliance.
  \section{PCI Compliance:}
    \subsection{Payment Card Industry Data Security Standard (PCI DSS):}
      The PCI DSS is a set of security standards that any organization transmitting or storing
      cardholder data must abide by. To be PCI compliant the organization must also secure validation
      of their compliance from the proper authorities.
      These twelve standards are grouped into six categories.
      \begin{enumerate}
        \item Build and Maintain a Secure Network
          \begin{itemize}
            \item Install and maintain a firewall configuration to protect cardholder data
            \item  Do not use vendor-supplied defaults for system passwords and other
security parameters
          \end{itemize}
        \item Protect Cardholder Data
          \begin{itemize}
            \item Protect stored cardholder data
            \item Encrypt transmission of cardholder data across open, public networks
          \end{itemize}
        \item Maintain a Vulnerability Management Program
          \begin{itemize}
            \item Protect all systems against malware and regularly update anti-virus
software or programs
            \item Develop and maintain secure systems and applications
          \end{itemize}
        \item Implement Strong Access Control Measures
          \begin{itemize}
            \item Restrict access to cardholder data by business need-to-know
            \item Identify and authenticate access to system components
            \item Restrict physical access to cardholder data
          \end{itemize}
        \item Regularly Monitor and Test Network
          \begin{itemize}
            \item Track and monitor all access to network resources and cardholder data
            \item Regularly test security systems and processes
          \end{itemize}
        \item Maintain an Information Security Policy
          \begin{itemize}
            \item  Maintain a policy that addresses information security for all personnel
          \end{itemize}
      \end{enumerate}
    Each of these requirements are further detailed by the PCI DSS to describe specific actions a PCI compliant organization must take.
  \subsection{Compliance Validation}
    Compliance validation is an involved process, requiring the participation of two entities.
    \subsubsection{Qualifid Security Assessor (QSA)}
      A QSA is an individual who has been certified by the PCI Security Standards Council.
      This individual can audit organizations for PCI compliance validation.
    \subsubsection{Internal Security Assessor (ISA)}
      An ISA is an individual who has been certified by the PCI Security Standards Council to perform
      PCI assessments for the organization they are representing. This process typically involves filling out
      the Self-Assessment Questionnaire.
  \section{Payment Methods:}
    \subsection{Third-Party Payment Processor vs. Merchant Account}
      \paragraph{Merchant Account/Direct Processor:}
        We have the option of being issued a merchant accout by a direct processor. A direct processor
        is typically a bank. In this circumstance the merchant account is a unique type of bank account.
        This offers the greatest degree of stability. Account holds or terminations are very unlikely.
        When holds or terminations occur customers experience an interruption of service.
        This additional stability comes at the expense of a thorough and expensive intial vetting process, as well
        as high monthly and annual fees. Methods of this type are only recommended with approximately \$5,000 - \$10,000
        in monthly transactions.
      \paragraph{Third-Party Payment Processor:}
        Our second option is establishing an account with a third-party payment processor. These processors having many
        merchants that they aggregate into a single merchant account with a direct processor. Their clients avoid
        the intense vetting process involved in establishing a merchant account. Third-party processors require a minimal initial vetting process,
        charge minimal fees, and hold month by month agreements. This offers the most flexibility and immediate profitability. Third-party processors conduct
        extensive ongoing vetting of their clients. This increases chances of account holds or terminations.
    \subsection{Payment Gateways}
        All credit card processing will require a payment gateway, a interface customers can use to make their payments.
        Few direct processors offer a payment gateway. Developing an in house gateway is another cost to consider with the direct processor option.
        Many third-party processors offer their own polished payment gateway.
    \subsection{Potential Processors:}
      All of the processors outlined here are PCI compliant.
      \subsubsection{CDGcommerce}
        \begin{wrapfigure}{l}{0.1\textwidth}
          \centering
          \includegraphics[width=0.1\textwidth]{cdgcommerce-logo}
        \end{wrapfigure}
        CDGcommerce is a merchant account provider, working as a middlde man between their clients and direct processors.
        Unlike most merchant account providers they do not charge setup, annual, or PCI compliaince fees. Additionally they
        offer two free payment gateway options. CDGcommerce charges a \$10.00 monthly support fee, and offers a \$15.00 month security service which includes
        \$100,000.00 in data breach insurance. CDG charges a 1.95\% + \$0.30 fee on most cards. On corporate, international, and premium cards this jumps to
        2.95\% + \$0.30. CDGcommerce's 24/7 customer service is highly rated, and the company holds an A+ with the Better Business Bureau.
      \subsubsection{Fattmerchant}
      \begin{wrapfigure}{l}{0.1\textwidth}
        \centering
        \includegraphics[width=0.1\textwidth]{fattmerchant}
      \end{wrapfigure}
        Fattmerchant is a mercant account provider. Like CDGcommerce it is not a direct processor, but an intermediary for them.
        Also like CDGcommerce, it does not charge setup, annual or PCI compliance fees. Fattmerchant offers 3 payment gateways at a monthly cost of \$7.95.
        Fattmerchant charges \$99.00 per month. Per transaction Fattmerchant charges a flat fee of \$0.15. Flattmerchants 24/7 customer service is highly rated
        and the company holds an A+ with the Better Business Bureau.
      \subsubsection{Paypal}
        \begin{wrapfigure}{l}{0.1\textwidth}
          \centering
          \includegraphics[width=0.1\textwidth]{paypal}
        \end{wrapfigure}
        Paypal is a third-party payment processor. Paypal is nearly ubiquitous as a payment processor on webstores. Paypal requires customers to have a paypal account to process a transaction.
        It is likely that many clients have a paypal account and are comfortable using paypals proprietary gateway. This trust and confidence
        could increase the use of our webstore. Paypal does not charge any setup, annual, monthly, or PCI copliance fees. Paypal charges 2.9\% + \$0.30 per transaction.
        Paypal does not have highly rated customer support. However, because of the popularity of its use, solutions to most problems are readily found online. Paypal holds a
        B with the Better Business Bureau.
      \newpage
      \subsubsection{Square}
        \begin{wrapfigure}{l}{0.1\textwidth}
          \centering
          \includegraphics[width=0.1\textwidth]{square}
        \end{wrapfigure}
        Square is a third-party payment proessor. Square is best known for its smartphone card readers. Square is popular, and most customers will be
        familiar with it. Square does not charge any setup, annual, monthly, or PCI compliance fees. Square offers many tools that we could leverage.
        This includes their proprietary payment gateway. Additionally Square offers a basic, mobile-compatible, online store that syncs automatically with in-person Square mediated payments.
        Square charges 2.9\% + \$0.30 per payment. Square has decent customer support, but it is not available 24/7. Square holds an A+ rating with the Better Business Bureau.

      \subsubsection{Comparison}
        \begin{center}
         \centering
         \begin{tabular}{||c c c c c c c||}
         \hline
         Processor & Type & Setup & Monthly & Transaction & 24/7 Support & Notes\\ [0.5ex]
         \hline\hline
         CDGcommerce & MA & \$0.00 & \$10.00 & 1.95\% - 2.95\% + \$0.30 & Yes & Offers data breach insurance up to \$100,000 for \$15/month\\
         Fattmerchant & MA & \$0.00 & \$99.00 & \$0.15 & Yes & Flat transaction fee\\
         Paypal & TPP & \$0.00 & \$0.00 & 2.9\% + \$0.30 & No & Extremely familiar to customers\\
         Square & TPP & \$0.00 & \$0.00 & 2.9\% +\$0.30 & No & Offers webstore service\\ [1ex]
         \hline
        \end{tabular}
        \caption{Comparison of various providers}
        \label{table:1}
      \end{center}

  \section{Recommendations:}
    \subsection{\$100,000/year}
      I recommend Fattmerchant at this annual gross sales profit.
      \paragraph{Assumptions:}
        \begin{itemize}
          \item We are not considering data breach insurance.
        \end{itemize}
      \paragraph{Justification:}
        Fattmerchant and CDGcommerce are both merchant accounts that are cheaper per transaction than
        either third-party processor and have no setups fees. Recall that merchant accouts are preferred for businesses. Their monthly fees are small compared to our sales volume. For these reasons
        we can dismiss Square or Paypal as serious options.
        Fattmerchant has a monthly fee of \$107 once a payment gateway service is added in.
        This is an initially higher cost than CDGcommerce. However CDGcommerce has a percentage base charge of all transactions.
        Using CDGcommerce would result in at least a 1.95\% loss of our sales volume, totalling \$1,950. In a year Fattmerchant would charge
        \$1164 more than CDGcommerce would from its \$10.00 monthly fees. As CDGcommerce also charges a higher flat-rate transaction fee than Fattmerchant,
        we can conclude that Fattmerchant is overall less expensive at this sales volume.
        In other respects Fattmerchant and CDGcommerce are very similar providers.
    \subsection{\$100,000,000/year}
      At this sales volume I still reccomend Fattmerchant.
      \paragraph{Justification:}
        The reasoning is identitical, but holds up more strongly here as 1.95\% of our sales volume is now \$19,500 making the disparity
        in cost between Fattmerchant and CDGcommerce more extreme.
  \section{Sources:}
    \begin{enumerate}
      \item https://www.merchantmaverick.com/best-online-credit-card-processing-companies/
      \item https://www.merchantmaverick.com/reviews/fattmerchant-review/
      \item https://www.merchantmaverick.com/reviews/paypal-review/
      \item https://www.merchantmaverick.com/reviews/square-review/
      \item http://www.onlinetech.com/resources/references/what-is-pci-compliance
      \item https://www.pcicomplianceguide.org/faq/
      \item https://www.bbb.org/en/us
    \end{enumerate}
\end{document}
