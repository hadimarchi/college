\documentclass{article}
\usepackage{graphicx}
\usepackage[T1]{fontenc}%
\usepackage[utf8]{inputenc}%
\usepackage{lmodern}%
\usepackage{textcomp}%
\usepackage{lastpage}%
\usepackage[export]{adjustbox}
\usepackage[top=1in, bottom=1.25in, left=1.00in, right=1.00in]{geometry}

\title{CS471 Midterm}
\author{Hal A. DiMarchi}

\begin{document}
\begin{titlepage}
  \normalsize
  \pagenumbering{gobble}
  \centering
  \vfill
  {\bfseries\Large
    CS471 Midterm \\
    Hal A. DiMarchi \\
    }
  \vfill
\end{titlepage}
\newpage
\section{Problem 1}
  You are part of a team charged with writing a library of mathematical functions
  involving floating-point values. Your team uses the Waterfall method.
  Write a complete set of requirements for the library.
  \subsection{Response}
    \subsubsection{Non-Functional Requirements}
      \begin{enumerate}
        \item The library will be written in Python 3.6.
        \item The library will be deployed to the Python Package Index for easy install with pip.
        \item All mathematical functions will be written with most efficient algorithms possible based on average case performance
        \item A user of the package already familiar with floating point math will
              be able to use the majority of system functions within two hours of self-guided training.
        \item The library will function on any system that is capable of using the Python 3.6 interpreter.
        \item The library will not use any third party packages beyond those that are part of the Python Standard Library.
        \item Any one function not accepting lists or dictionaries of arguments will take no more than half a second to execute.
        \item The library will utilize multiple threads, if possible, when performing operations on lists of values with lengths of at least 100.
      \end{enumerate}
    \subsubsection{Functional Requirements}
      \begin{enumerate}
        \item Provided functions
        \begin{itemize}
          \item The library will provide functions for all basic arithmetic operations on floating point values.
          \item The library will provide functions for all trigometric operations on floating point values.
          \item The library will provide functions for all hyperbolic operations on floating point values.
          \item The library will provide functions for summing over a list and finding the product of a list of values.
          \item The library will provide functions for logarithmic operations at any base.
          \item The library will provide functions for rounding a floating point value to any decimal precision or to the nearest integer.
          \item The library will store, and provide functionality to return a text representation of, the last twenty function calls.
        \end{itemize}
        \item The library will support both negative and positive floating point values and operations on them.
        \item When a function is passed an innapropriate type it will raise a TypeError.
        \item When a function is passed a value outside of the range of values it can operate on it will raise a ValueError.
        \item The library will treat integers as floating point values with a decimal value of 0.
        \item The library will by default calculate and store floating point values at a
              decimal place accuracy of twenty, except when the exact decimal value can be represented
              in fewer than twenty digits.
      \end{enumerate}

\section{Problem 2}
  You are employed as a software developer for a company that uses the Scrum methodology
  on all software development projects. As part of a major expansion, your company
  is hiring a number of experienced software developers, none of who have ever used
  Scrum before, but all of whom have completed a number of projects using Kanban.

  Write a guide for the newly hired developers explaining how Scrum works—comparing
  and contrasting with Kanban as appropriate—to get them up to speed quickly.
  \subsection{Response}
  Scrum, as an Agile method, contains many of the components of Kanban.
  You will be familiar with the idea of a kanban board to visualize workflow and split user stories into several stages.
  Scrum shares the components of user stories and a visualization board,
  but the board tends to be more simple with only
  four stages; backlog, not started, in progress, and done. This stages are typically
  not divided into active and done. User stories themselves, and the process of deciding
  if an item is overall 'done' are essentially identical.

  Like Kanban, Scrum involves daily meetings. These meetings are often referred to as
  Scrums, Scrumming, or just Scrum. Scrum is led by a Scrum master, very similar to
  the Team project manager from Kanban. Each meeting involves each team member discussing
  what they have done since the last meeting, and what they will be working on that day,
  along with problems they've run into and anything else the team needs to be aware of.

  While Scrum also prioritizes incremental development it approaches development overall
  in a much more structured way. Work on a Scrum team is completed in a series of 'sprints'
  that typically occurr over a two week period.

  A sprint is begun with a planning meeting in which items from the backlog are selected to be worked on during that sprint and moved
  into the 'not started' section of the Scrum board. One member of these meetings is known as the 'Product Owner'.
  This person represents the customer who is asking the team to build their product.
  Ideally this truly is the customer, or a member of their organization,
  but often it is simply the Scrum master. This person is the arbiter of what is even in the backlog,
  and what the priority of each of those items is.

  A sprint then continues for a set period of time in which items moved out of the backlog
  are selected and worked on by individual team members. Team members typically select one user story
  at a time, and select a new one once the item has been completed and validated.

  At the end of the sprint team members conduct a sprint review in which they meet with the
  product owner and often developers from other teams within their organization to present and
  demonstrate what they've implemented during the sprint. This allows for the team and product
  owner to be on the same page before another planning meeting occurs.

  Sprints make Scrum far more deadline and planning oriented than Kanban.


\section{Problem 3}
In 2011 the Municipality of Anchorage began a project to develop a software system
to handle payroll and other city government functions.
The new system, developed by SAP, replaces the old PeopleSoft system.
\subsection{Response}
  \subsubsection{PeopleSoft System}
    The PeopleSoft software suite is an 'enterprise resource planning' (ERP) system
    meaning that it provides tools for organizations to manage and track their resources
    including human resources and funding. PeopleSoft provided financial and supply chain
    management in a module known as 'Financials and Supply Chain Management'(FSCM). This included
    functionality for handling invoices, paychecks, budget and labor contracts and tax information.
  \subsubsection{Replacement Reasoning}
  The third party contractor, Gartner, was consulted to conduct an ERP review. Their review
  supported an upgrade to a new PeopleSoft implementation citing technical issues with
  the decade old implementation. This review supported this upgrade over a replacement with SAP
  ERP software because it projected that, despite an initially lower cost, SAP's software
  and the process of implementing it as the city of Anchorage's ERP software would eventually
  cost more. Anchorage ignored this second piece of reasoning. A review done by contractor Den
  Howlett suggests this decision was made because of the lower cost presented by SAP. It is thought
  that Anchorage ignored Gartner's analysis because of its purely cost/benefit analysis approach to the
  review, which lacked an analysis of the capabilities of Anchorage to manage this process.
  \subsubsection{SAP and their Software}
    SAP is a software development company based in Germany. It primarily develops enterprise software,
    including ERP systems. SAP's software includes software for managing Operations,
    Financials (Financial Accounting, Management Accounting, Financial Supply Chain Management), and
    Human Capital Management (Training, Payroll, e-Recruiting). Anchorage was primarily interested in
    handling invoices, paychecks, budget and labor contracts and tax information, which SAP's ERP software was capable of.
  \subsubsection{Timeline}
  \begin{itemize}
    \item 2011: SAP project begins with a 10.6 million dollar budget and a launch date in 2012.
    \item 2012: No launch.
    \item 2013: Additional funding is granted to the project and the launch date is pushed back.
    \item 2014: Launch is delayed again. Anchorage assembly considers and approves an external audit of the project.
    \item 1st half of 2015: The reviews are finished. Problems with management, documentation, staff, and lack of a well-defined project plan are discovered.
                Reviews also conclude the project should continue.
    \item 2nd half of 2015: Ethan Berkowitz replaces Dan Sullivan as mayor, plans to pause SAP project, but it continues.
    \item 2016: Anchorage highers a former IBM executive as project manager.
    \item 2017: Project has cost \$81 million at this point. Despite strong criticism several million more are given and SAP launches.
    \item 2018: SAP has overpayed and underpayed employees thousands of times since its launch. That said, the rate at which it does so is dropping.
  \end{itemize}
  \subsubsection{Current Status}
    Currently the SAP software is still being used by Anchorage government. In June it received 155 reports of payroll
    error, down from 1,200 in its first month of operation.
  \subsubsection{Largest Problem}
    The largest problem for the project seems to have been the lack of a clear blueprint
    or specified set of software requirements. The team implementing the SAP
    software was using a Waterfall approach to project management. This approach
    relies on having a fully fleshed out design before taking any steps to further the
    project. In addition to the lack of software requirements, no testing strategy was present.
    In a system with particularly costly consequences for error this step was particularly important.
    All in all the team failed to effectively use any project management techniques and so their
    attempts crumbled and stagnated for years.

\section{Problem 4}
  We need to develop a checklist to perform software inspections later in the semester.
  Research the topic and create a 10-question checklist that would be appropriate for class assignments at the level of CS 311.
  \subsection{Response}
  \begin{enumerate}
    \item Are variables that go unchanged declared as const?
    \item Are variables named in such a way that their purpose is easy to understand?
    \item Are functions that want to modify a paramater taking it in by reference?
    \item Do any array access operations access an index outside of the array?
    \item Are all functions, classes, and variables declared before they are used?
    \item Are all templated functions or classes defined in header files?
    \item Are all types and functions from the standard library prefaced with
          'std::' or in files with an appropriate 'using std::"thing"' statement?
    \item Can your code be easily understood by a peer, or is further documentation necessary?
    \item Do any of your classes define a copy or move constructor or a destructor? If any are defined you likely want to define all three.
    \item Are all pointers set to null when the object they are pointing at is destroyed?
  \end{enumerate}









\end{document}
