\documentclass{article}
\usepackage{graphicx}
\usepackage[T1]{fontenc}%
\usepackage[utf8]{inputenc}%
\usepackage{lmodern}%
\usepackage{textcomp}%
\usepackage{lastpage}%
\usepackage[export]{adjustbox}
\usepackage{amsmath}
\usepackage[top=1in, bottom=1.25in, left=1.00in, right=1.00in]{geometry}

\begin{document}%
\normalsize%
\begin{titlepage}
  \pagenumbering{gobble}
  \centering
  \vfill
  {\bfseries\Large
      Algorithms Homework 4\\
      H. DiMarchi\\
  }
  \vfill
\end{titlepage}
\pagenumbering{gobble}
\newpage
\section{Chapter 3}
  \subsection{Section 2}
    \subsubsection{Problem 8}
      \paragraph{a.}
        Start by initializing a count of 0.
        Iterate through the string. Check each character. If it is an A, iterate
        through the rest of the string, checking each character: If the character
        is a B, increment your count of substrings that start with A and end with B.
        Continue until you have finished the top level iteration through the string.
      \paragraph{b.}
        Start by initializing a count of 0, and an increment\_amount of 1.
        Iterate through the string, checking each character. If the character
        is a B, increment your count of substrings that begin with A and end with begin
        by your increment amount. If the character is an A,
        increment your increment count by 1: By doing this you will account for
        each substring that could be constructed from a particular B after that A.
        Continue this until you have finished iterating through the string.

        This is a more efficient algorithm because it requires you to iterate
        through the string only once.
    \section{Chapter 4}
      \subsection{Section 5}
        \subsubsection{Problem 13}
        Iterate through matrix by length of rows such that each item you touch on
        begins a row. For each item, check if that item is greater than or equal to the value you are
        searching for. If that item is equal to the value, return that index. If that item is greater than the value look through preceding row from the
        second element of that row to the end of that row for the value you are searching for until you find it,
        then return the index of the value searched for.
        If that item is neither go to the first element of the next row and make the same check.
        If you arrive at the last row, and its first value is less than the value searched for
        , iterate through that row for the value.
        \newline
        Value searched for: 7
        \[
        \begin{bmatrix}
        1 & 2 & 3 & 4 \\
        5 & 6 & 7 & 8 \\
        9 & 10 & 11 & 12 \\
        13 & 14 & 15 & 16
        \end{bmatrix}
        \]
        Items Visited: 1 -> 5 -> 9 -> 6 -> 7 -> done

    \section{Chapter 5}
      \subsection{Section 1}
        \subsubsection{Problem 5}
          \paragraph{a.}
          \paragraph{b.}
          \paragraph{c.}










\end{document}
