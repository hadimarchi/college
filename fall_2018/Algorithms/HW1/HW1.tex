\documentclass{article}
\title{Analysis of Algorithms: Assignment 1}
\author{Hal DiMarchi}
\usepackage{listings}
\usepackage{amsmath}
\begin{document}
\pagenumbering{gobble}
\maketitle
\newpage
\pagenumbering{arabic}
\section{Section 1.3}
  \subsection{Problem 1}
    \subsubsection{b. Is this algorithm stable?}
      \paragraph{Answer:}
        No. As you can see in the innermost block of code, if A[i] >= A[j] (aka not less than)
        Count[i] is incremented, meaning A[i] will be moved ahead of A[j] when these elements are equal.
        This reverses their relative order, making the algorithm unstable.
    \subsection{c. Is it in place?}
      \paragraph{Answer:}
        No. A new array (S) is returned from the function, thus the algorithm violates the definition of in-place.
\section{Section 1.4}
  \subsection{Problem 3}
    \subsubsection{Create a simple algorithm for the string-matching problem}
      \paragraph{Answer:}
        Create a list of all the substrings in your string that are of the same length as your pattern.
        Simultaneously create a list of integers that denote the index at which each substring begins in the parent
        string.
        Compare your pattern to each of these substrings. If the substring and the pattern match, return, or push, the corresponding
        starting index of the substring.
\section{Section 2.1}
  \subsection{Problem 7}
         Gaussian elimination, the classic algorithm for solving systems of n linear
         equations in n unknowns, requires about $1/3(n^3)$ multiplications, which is the
         algorithm’s basic operation.
     \subsubsection{a. How much longer should you expect Gaussian elimination to work on a
                    system of 1000 equations versus a system of 500 equations}
        \paragraph{Answer:}
          \begin{align}
            G(500)=1/3(500^3)=41666666.6667 \\
            G(1000)=1/3(1000^3)=333333333.333 \\
            G(1000)\div G(500) = 8
          \end{align}
          I should expect it to take eight times longer on a system of 1000 equations versus a system of 500 equations.


  \subsection{Problem 8}
        For each of the following functions, indicate how much the function’s value
        will change if its argument is increased fourfold.
    \subsubsection{a. $\log_{2}(n)$}
          \paragraph{Answer:}
              $\log_{2}(4n) = log_{2}(4) + log_2(n) = 2 + log_2(n)$ \\
              The value will increase by 2.
    \subsubsection{b. $\sqrt{n}$}
  \subsection{Problem 9}
\section{Section 2.2}
  \subsection{Problem 2}
\section{Section 2.3}
  \subsection{Problem 1}
  \subsection{Problem 5}
\section{Section 2.4}
  \subsection{Problem 1}
  \subsection{Problem 4}

\end{document}
